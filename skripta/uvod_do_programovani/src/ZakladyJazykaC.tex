\chap {Základy jazyka C}
Jazyk C je jedním z nejstarších a nejvlivnějších programovacích jazyků, který byl vyvinut v 70. letech 20. století v Bellových laboratořích. Tento jazyk se stal základem pro mnoho moderních programovacích jazyků, jako je C++, JavaScript a dokonce i některé jazyky vyšších úrovní jako Java a Python. 

Vlastnosti jazyka C:
\begitems
* {\bf Nízká úroveň abstrakce} - Jazyk C poskytuje abstrakci na použitým hardwarem (na úrovni jazyka programátora nezajímá jaký procesor programuje), ale zároveň umožňuje přímí přístup k paměti. 
* {\bf Vysoká optimalizace} - Díky své jednoduchosti je možné využít vysokou míru optimalizace výsledného programu. Díky tomu poskytují výsledné programy vysoký výkon.
* {\bf Jednoduchost a přímočarost} - Syntaxe jazyka C je jednoduchá a přímočará (v porovnání s jazyky vyšší úrovně). 
\enditems

Použití jazyka C:
\begitems
* {\bf Systémové programování} -  C je široce používán pro vývoj operačních systémů (například UNIX), ovladačů zařízení, vestavěných systémů a dalších nízkoúrovňových aplikací, kde je vyžadován přímý přístup k hardwaru.
* {\bf Aplikace s vysokým výkonem} - Vzhledem k tomu, že C poskytuje velkou kontrolu nad výkonem a pamětí, je ideální pro vývoj aplikací, kde jsou kladeny vysoké nároky na výkon, jako jsou grafické programy, hry, simulace nebo vědecké výpočty.
* {\bf Embedded systémy} - C je jazykem první volby pro vývoj embedded (vestavěných) systémů, kde je efektivní využívání paměti a výkonu klíčové. Příkladem vestavěných systémů je například chytrý domácí spotřebič (robotický vysavač, pračka, ...)
\enditems

\sec {Příprava prostředí pro vývoj v jazyce C}
\secc {MS Windows}
Na systémech MS Windows je nejjednodušší způsob pro nastavení prostředí pro vývoj v jazyce C použít program {\bf msys2} (\url {https://www.msys2.org/}). Tento program vytvoří strukturu nástrojů, které jsou k dispozici na systémech Linux. Po instalaci je třeba nejprve třeba nastavit systém, aby mohl vyhledat nástroje, které se do počítače nainstalovali. K tomu je nutné upravit {\bf proměnnou prostředí PATH}, do které je třeba přidat dvě nové cesty. Proměnné prostředí je možně upravovat zadáním příkazu do nabýdky start: {\it upravit proměnné prostředí systému}

\vskip 5mm
\picw=.7\hsize \centerline{\inspic {\imgpath setup_environment_variables.png} }\nobreak\medskip
\caption/f Nastavení proměnné prostředí Path

Poté se otevře okno pro nastavení proměnných prostředí ve kterém je tlačítko {\bf Proměnné prostředí}, které otevře okno s proměnnými prostředí, které jsou v systému vytvořené. V sekce {\bf Uživatelské proměnné} se nachází výčet proměnných prostředí pro aktuálně přihlášeného uživatele a mezi nimi se nachází proměnné {\bf Path}. Po jejím označení je třeba stisknout tlačítko upravit. Následně by se mělo otevřít okno pro úpravu hodnot v této proměnné. Nyní je nutné přidat dvě nové cesty pomocí tlačítka {\bf Nový} a nastavit jejich hodnotu na: 
\begtt 
C:\msys2\usr\bin 
\endtt 
 
\begtt 
C:\msys2\mingw64\bin 
\endtt 

Následně stačí jen vše uložit stiskem tlačítka OK a cesty k nástrojům systému msys2 by měly být nastavené.

\vskip 5mm
\picw=.9\hsize \centerline{\inspic {\imgpath environment_variables.png} }\nobreak\medskip
\caption/f Okno pro nastavení proměnných prostředí

Nově nastavené cesty je možné věřit otevřením příkazové řádky pomocí zadání příkazu {\it cmd} do nabídky start a zadat příkaz, který by měl provést aktualizaci systému msys2:
\begtt
$ pacman -Syu
\endtt

(Prosím nekopírujte příkaz se znakem \$, ten značí že se jedná o terminálový vstup)\par
V případě, že jsou cesty správně nastavené by se měl spustit balíčkový správce {\it pacman}, který slouží k instalaci a aktualizaci programů do prostředí msys2 a požádá vás o potvrzení spuštění instalace aktualizací:

\vskip 5mm
\picw=.9\hsize \centerline{\inspic {\imgpath msys2_update.png} }\nobreak\medskip
\caption/f Okno pro nastavení proměnných prostředí

Aktualizace a jiné instalace se potvrdí prostým stiskem klávesy {\it Enter}.

Po aktualizaci prostředí msys2 je potřeba nainstalovat kompilátor pro jazyk C {\bf gcc}, který má za úkol převést zdrojový k na spustitelný a nástroj pro automatizaci překladu {\bf make}. To se provede zadáním příkazu:
\begtt
$ pacman -S gcc make
\endtt

Po této instalaci je prostředí pro vývoj programů vjazyce C na systému MS Windows připravené k použití, ale je potřeba nainstalovat tzv. IDE (Integrated development environment), zjednodušeně řečeno textový editor, který obsahuje nástroje pro zjednodušení psaní zdrojových kódu. Nejjednodušší IDE je {\bf Visual studio code} (\url {https://code.visualstudio.com/}). Jeho instalace je jednoduchá a přímočará. 

\vskip 5mm
\picw=.9\hsize \centerline{\inspic {\imgpath vs_code.png} }\nobreak\medskip
\caption/f Visual studio code

\sec {Syntaxe jazyka C}
Jazyk C je založen na několika základních stavebních blocích, mezi něž patří:
\begitems
* {\bf Proměnné}
* {\bf Podmínky}
* {\bf Cykly}
* {\bf Funkce}
* {\bf Preprocesor}
\enditems

\secc {Základní struktura projektu v jazyce C}




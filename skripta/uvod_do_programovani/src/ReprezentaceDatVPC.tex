\chap {Reprezentace dat v paměti PC}
Počítače jsou postaveny na elektronických obvodech, které pracují se dvěma základními stavy:

\begitems
* {\bf Zapnuto} - proud prochází, což odpovídá logické hodnotě 1.
* {\bf Vypnuto} - proud neprochází, což odpovídá logické hodnotě 0.
\enditems

Tento dvoustavový systém je jednoduchý, levný na výrobu a velmi spolehlivý. Složitější systémy (například desetistavové) by vyžadovaly přesnější měření a byly by náchylnější k chybám kvůli šumu nebo odchylkám v signálu. Kombinací těchto nul a jedniček lze vytvořit složitější struktury, které reprezentují různé typy informací. 

\sec {Jednotky paměti}
\secc {Bit}
{\bf Bit} (zkratka z "binary digit") je základní jednotka informace v počítači. Bit může mít pouze dvě hodnoty: 0 nebo 1. 

\secc {Bajt}
Pokud se spojí dohromady 8 bitů, lze získat 256 kombinací hodnot jednotlivých bitů, které mohou kódovat hodnoty od 0 do 255. Spojení 8 bitů se říká {\bf Bajt}. Pro reprezentaci paměťové kapacity úložných zařízení se běžně používají násobky bajtů:

\begitems
* {\bf Kilobajt} - 1024 bajtů
* {\bf Megabajt} - 1024 kilobajtů
* {\bf Gigabajt} - 1024 megabajtů
* ...
\enditems

\sec {Kódování dat v binární soustavě}
Způsobů jak kódovat data v binární soustavě je více a každý z nich se využívá ve specifických případech. V paměti počítačů se využívá kódování dat založené na číselných soustavách. Číselná soustava je způsob, jakým zapisujeme a reprezentujeme čísla pomocí určitých symbolů (číslic). Každá číselná soustava má {\bf základ} (například 2 pro dvojkovou nebo 10 pro desítkovou), který určuje, kolik různých číslic se používá a jak se čísla skládají. {\bf Číselný řád} označuje pozici číslice v čísle, která určuje její hodnotu podle mocniny základu číselné soustavy. Například v desítkové soustavě jsou číselné řády:
\begitems 
* {\bf Jednotky} - $10^0 = 1$
* {\bf Desítky} - $10^1 = 10$
* {\bf Stovky} - $10^2 = 100$
* {\bf Tisíce} - $10^3 = 1000$
* ...
\enditems

Každé číslo pak lze zapsat v polynomiálním tvaru. {\bf Polynomiální tvar} čísla je způsob, jak zapsat číslo jako součet jednotlivých číslic vynásobených jejich hodnotou podle jejich pozice (řádu). Například:
$$
123_{(10)} = (10^2 \cdot 1) + (10^1 \cdot 2) + (10^0 \cdot 3)
$$

Stejným způsobem lze definovat číselné řády také v binární soustavě, pouze jako základ zvolíme číslo 2:

\begitems 
* {\bf Jednotky} - $2^0 = 1$
* {\bf Desítky} - $2^1 = 2$
* {\bf Stovky} - $2^2 = 4$
* {\bf Tisíce} - $2^3 = 8$
* ...
\enditems

Stejným způsobem pak můžeme zakódovanou hodnotu v dvojkové soustavě převést na hodnotu v desítkové soustavě:

$$
1011_{(2)} = (2^3 \cdot 1) + (2^2 \cdot 0) + (2^1 \cdot 1) + (2^0 \cdot 1) = 8 + 0 + 2 + 1 = 11_{(10)}
$$


